\documentclass{beamer}

\title{How to define bird flocks}
\author{Louis von Leitner}
\institute{Tohoku University, Center for Data driven Science and Artifical Intelligence}
\begin{document}

\frame{\titlepage}

\section{Introduction}
\begin{frame}
    \frametitle{What are flocks of birds?}
    \begin{definition}
        Flocks of birds are groups of individuals that share similar characteristics like 
        position, direction and speed.
    \end{definition}
\end{frame}

\begin{frame}
    \frametitle{How to define Flocks mathematically?}
    \begin{block}{Problem}<1->
        Definitions of bird flocks are very vague in literature, because they are a complex concept.
    \end{block}
    \vspace{3pt}
    \begin{block}{Solution}<2->
    Instead of giving a definition for flocks, we give a definition for the \textbf{coherence} of 
    birds and derive a definition for \textbf{coherence} of flocks.\\
    This \textbf{coherence} concept gives us a mathematical parameter of \textit{flockiness} with which 
    it will be easier to define flocks.
    \end{block}
\end{frame}


\begin{frame}
    \frametitle{Coherence}
    \begin{definition}
        Let $i$ be a bird in a set of birds.
        \begin{itemize}
        \item<2-> The \textbf{position} of this bird $x_i \in \mathbb{R}^3$ is given in cartesian coordinates.
        \item<3-> Its \textbf{velocity} is $v_i \in \mathbb{R}^3$.
        \end{itemize}
    \end{definition}
\end{frame}

\section{Coherence}
\begin{frame}
    \frametitle{Simple Coherence}
    \begin{definition}
        Let $i, j$ be two birds.\\
        The \textbf{simple coherence} between the birds is given by
        \begin{center}
            $C_{i, j} = \underbrace{\frac{1}{||x_i - x_j||}}_{\textit{inverse distance}} \cdot 
            \underbrace{\frac{v_i \cdot v_j}{||v_i|| \cdot ||v_j||}}_{\textit{directional correlation}}$
        \end{center}
    \end{definition}
    \begin{block}{Remark}<2->
        \begin{itemize}
            \item As two birds cannot physically be at the same position, $x_i \neq x_j \Rightarrow ||x_i - x_j|| \neq 0$
            \item As birds are not stationary, because if they were, they would fall from the sky, $||v_i||, ||v_j|| \neq 0$
        \end{itemize}        
    \end{block}
\end{frame}

\begin{frame}
    \frametitle{Logarithmic Coherence}
    \begin{definition}
        The \textbf{logarithmic coherence} between birds $i, j$ is given by
        \begin{center}
            $L_{i, j} := \log(1 + \frac{1}{||x_i - x_j||}) \cdot \frac{v_i \cdot v_j}{||v_i|| \cdot ||v_j||}$
        \end{center}
    \end{definition}
\end{frame}

\begin{frame}
    \frametitle{Coherence}
    \begin{corollary}
        Let $i, j$ be birds from a set of birds.
        \begin{enumerate}
            \item If the birds are flying in the same direction, coherence is positive.
            \item If the birds are flying in directions more than 90 degrees offset, the coherence is negative.
            \item If the birds are close to one another, $|C_{i, j}|$ is bigger.
            \item If the birds are far from one another, $|C_{i, j}|$ is smaller.
        \end{enumerate}
    \end{corollary}
\end{frame}

\begin{frame}
    \begin{minipage}[t]{0.45\linewidth}
        \begin{figure}
            \centering
            \includegraphics[width=1.25\linewidth]{figures/coherence_simple.png}
            \label{fig:coherence_simple}
            \caption{Coherence Heatmap using \textbf{simple coherence}}
        \end{figure}
    \end{minipage}%
    \begin{minipage}[t]{0.45\linewidth}
        \begin{figure}
            \centering
            \includegraphics[width=1.25\linewidth]{figures/coherence_with_log.png}
            \label{fig:coherence_log}
            \caption{Coherence Heatmap using \textbf{logarithmic coherence}} 
        \end{figure}
    \end{minipage}
\end{frame}

\begin{frame}
    \begin{minipage}[t]{0.45\linewidth}
        \centering
        \begin{figure}
            \centering
            \includegraphics[width=1.25\linewidth]{figures/coherence_from_epicenter.png}
            \label{fig:coherence_simple_epicetner}
            \caption{Coherence Heatmap around epicenter using \textbf{simple coherence}}
        \end{figure}
    \end{minipage}%
    \begin{minipage}[t]{0.45\linewidth}
        \centering
        \begin{figure}
            \centering
            \includegraphics[width=1.25\linewidth]{figures/logarithmic_coherence_map_epicenter.png}
            \label{fig:coherence_log_epicenter}
            \caption{Coherence Heatmap around epicenter using \textbf{logarithmic coherence}} 
        \end{figure}
    \end{minipage}
\end{frame}

\begin{frame}
    \frametitle{Pros and cons of this method}
    \begin{minipage}[t]{0.45\linewidth}
       \begin{center}
        \textbf{+}
       \end{center} 
       \begin{enumerate}
        \item captures direction and space
        \item direction is more important than space
       \end{enumerate}
    \end{minipage}%
    \begin{minipage}[t]{0.45\linewidth}
       \begin{center}
            \textbf{-}
       \end{center} 
       \begin{enumerate}
        \item direction and space are not equally important, how to quantify that?
        \item other birds behind or in front a bird should be respected more than birds to the side or under
        \item Computationally inefficient (n number of birds)\\
        $\mathcal{O}(n)$ per bird $\rightarrow \mathcal{O}(n^2)$ in total\\
        with a lot of square roots
       \end{enumerate}
    \end{minipage}
\end{frame}

\end{document}